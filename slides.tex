% !TeX program = lualatex

% Fix colors
% https://github.com/tudace/tuda_latex_templates/issues/285#issuecomment-785017118
\PassOptionsToPackage{RGB}{xcolor}

\documentclass[
	english,%globale Übergabe der Hauptsprache
	aspectratio=169,%Beamer eigene Option zum Umschalten des Formates
	color={accentcolor=1b},
	logo=true,%Kein Logo auf Folgeseiten
	colorframetitle=true,%Akzentfarbe auch im Frametitle
%	logofile=example-image, %Falls die Logo Dateien nicht vorliegen
	authorontitle=true,
	usepdftitle=false,
%	pdfa=false,
	]{tudabeamer}
\usepackage[main=english]{babel}

% Der folgende Block ist nur bei pdfTeX auf Versionen vor April 2018 notwendig
\usepackage{iftex}
\ifPDFTeX
\usepackage[utf8]{inputenc}%kompatibilität mit TeX Versionen vor April 2018
\fi

\usepackage{hyperref}

\hypersetup{
	pdftitle={Your Thesis Title},
	pdfauthor={Your Name}
}

\usepackage{caption}
\captionsetup{font=scriptsize,labelfont=scriptsize}
\setbeamerfont{caption}{size=\scriptsize}

\usepackage{tabularx}

% tikz
\usepackage{tikz}
\usepackage{pgfplotstable}
\usepackage{pgfplots}
\usetikzlibrary{shapes,arrows,positioning,fit,calc}
\usepgfplotslibrary{units}
\pgfplotsset{compat = 1.3}

\usepackage{subcaption}

\usepackage[fleqn]{mathtools}
\usepackage{siunitx}

\usepackage{amsmath}  % for \hookrightarrow
\usepackage{xcolor}   % for \textcolor
\usepackage{listings}
\lstset{
	basicstyle=\tiny\ttfamily,
	columns=fullflexible,
	frame=single,
	breaklines=true,
%	postbreak=\mbox{\textcolor{red}{$\hookrightarrow$}\space},
	postbreak=\mbox{$\hookrightarrow$\space},
}

% Footnote style
\renewcommand*{\thefootnote}{[\arabic{footnote}]}
\newcommand{\ccite}[1]{\footnote{\tiny\textcolor{TUDa-0b}{#1}}}
\newcommand{\cciteframe}[1]{\footnote[frame]{\tiny\textcolor{TUDa-0b}{#1}}}

% PDF notes for presentation mode
\usepackage{pgfpages}
\setbeameroption{show notes}
% TODO: Comment the following line out to get a "normal" PDF without notes
\setbeameroption{show notes on second screen=right}
% TODO: Check out pdfpc to present your PDF slides (with notes): https://pdfpc.github.io/

% Load CSV files
\usepackage{csvsimple}

% Diagonal fractions
% https://tex.stackexchange.com/questions/3372/how-do-i-typeset-arbitrary-fractions-like-the-standard-symbol-for-5-%C2%BD
\usepackage{xfrac}

% More font sizes
% https://tex.stackexchange.com/questions/48276/latex-specify-font-point-size
\usepackage{moresize}

%Makros für Formatierungen der Doku
%Im Allgemeinen nicht notwendig!
\let\code\texttt

\title{Your Title}
\subtitle{Master-Thesis (final presentation)}
\author[Your Name]{Your Name \normalfont(\href{mailto:your.name@stud.tu-darmstadt.de}{your.name@stud.tu-darmstadt.de})\\
Supervisor: Another name
}

% Removed department and institute on purpose
%\department{Department 18} % Remove department everywhere
\institute[]{Real-Time Systems Lab} % Removed institute only in footline

%Fremdlogo
%Logo Macro mit Sternchen skaliert automatisch, sodass das Logo in die Fußzeile passt
\logo*{\includegraphics{figures/es_logo_gross.jpg}}

% Da das Bild frei wählbar nach Breite und/oder Höhe skaliert werden kann, werden \width/\height entsprechend gesetzt. So kann die Fläche optimal gefüllt werden.
%Sternchenversion skaliert automatisch und beschneidet das Bild, um die Fläche zu füllen.
%\titlegraphic*{\includegraphics{example-image}}

\date{\today}


\begin{document}
\maketitle

%
% Introduction
%

\section{Introduction}

\begin{frame}{Motivation}
	\framesubtitle{Why should anyone be interested in this topic?}
	\label{motivation}
	
	\begin{columns}[onlytextwidth,c]
		\column{.5\linewidth}
		\begin{itemize}
			\item Words
			\begin{itemize}
				\item More
				\item Words
			\end{itemize}
			\vspace{1em}
		
			\item How to write \textit{more} words?
			
			\item \textbf{You can ask your supervisor for some example slides built with this template.}
		\end{itemize}
	
		\column{.5\linewidth}
		\vspace{-0.5em}
		You can write text in here, too.
	\end{columns}
\end{frame}

\begin{frame}{What exactly is your topic?}
	\framesubtitle{Example}
	\label{what-is-this}
	
	\begin{itemize}
		\item Do not forget to include some nice graphics.
	\end{itemize}
\end{frame}
\note[itemize]{
	\item You can write noted down here.
	\item Use a PDF presentation program like \url{https://pdfpc.github.io/} to get a split-screen while presenting.
}

\begin{frame}{Problem Description}
	\framesubtitle{What is the goal of my thesis?}
	\label{goal}
\end{frame}

%
% Main part
%

\section{Main part}

\begin{frame}{A Title}
	\framesubtitle{Why build a new XY framework?}
	\label{framework}
	
	\begin{itemize}
		\item \textcolor{TUDa-0c}{Why chose this color? Well $\ldots$}
	\end{itemize}
\end{frame}
\note[itemize]{
	\item TODO
}

\begin{frame}{Evaluation Research Questions}
	\framesubtitle{How to evaluate the algorithm?}
	\label{eval-research-questions}
	
	\vspace{3.5em}

	\begin{columns}[onlytextwidth,c]
		\column{1.0\linewidth}
	
		\begin{itemize}
			\itemsep1.5em
			\item[\textcolor{black}{\textbf{RQ1}}] How does \textit{XY} compare against ZZ in terms of \textbf{BB quality}?
	
			\item[\textcolor{black}{\textbf{RQ2}}] How large is the \textbf{performance gain} of \textit{XY} compared to optimal approaches if we reduce the adherence to optimality step by step?
			
			\item[\textcolor{black}{\textbf{RQ3}}] How does \textit{XY} compare against another \textbf{heuristic approach} taken from literature?
		\end{itemize}

	\end{columns}
	
\end{frame}
\note[itemize]{
	\item More notes for you
}

\begin{frame}{Evaluation Setup}
	\framesubtitle{$\ldots$ is it really faster? (1) - Setup}
	\label{results-setup}
	
\end{frame}
\note[itemize]{
	\item TODO
}

\begin{frame}{Evaluation Results}
	\framesubtitle{$\ldots$ is it really faster? (2) - Small scenarios}
	\label{results-1}
	
\end{frame}
\note[itemize]{
	\item TODO
}

\begin{frame}{Evaluation Results}
	\framesubtitle{$\ldots$ is it really faster? (3) - Large scenarios}
	\label{results-2}
	
	\end{frame}
\note[itemize]{
	\item TODO
}

\begin{frame}[fragile]{Conclusion}
	\framesubtitle{What goals/insights were achieved?}
	\label{conclusion}
	
\end{frame}
\note[itemize]{
	\item This seems to be important
}

\section{Work to do}

\begin{frame}{Future Work}
	\framesubtitle{Further ideas to implement/evaluate}
	\label{future-work}
	\vspace{2em}
	
\end{frame}

\begin{frame}[c]{}
	\label{thank-you}
	\centering
	\vspace{3em}
	\LARGE
	Thank you for your attention!\\
	Any questions?
\end{frame}

%
% Backup slides starting here
%

\begin{frame}{Backup Slide - Increasing Problem Size}
	\framesubtitle{A subtitle}
	\label{problem-size}
	
\end{frame}
\note[itemize]{
	\item Please notice
}

\end{document}
