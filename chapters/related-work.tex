%!TEX root = ../thesis-main.tex

\chapter{Verwandte Arbeiten}\label{chapter:related-work}

Im Prinzip geht es darum ähnliche und verwandte Ansätze herauszusuchen und diese Zusammenhänge hier zu erläutern (Siehe Referenzen). Es können allerdings auch andere Anwendungsgebiete des Ansatzes, außerhalb des eigenen Szenarios an dieser Stelle aufgeführt werden.
Dazu gehört die fremde Arbeit zusammenzufassen (1-2 Sätze) und sie der eigenen Arbeit gegenüberzustellen (kritisch!).
Vor allem bei der kritischen Gegenüberstellung sollte man sich auf die Unterschiede zum eigenen Ansatz fokussieren.
Dabei geht es aber nicht darum, sich auf negative Aspekte zu konzentrieren, sondern auch seinem eigenen Ansatz gegenüber kritisch zu sein.
Oft ist es ohnehin der Fall, dass man in gewissen Teilaspekten bessere Ergebnisse erzielt und in anderen dafür Nachteile gegenüber anderen Ansätzen hat.