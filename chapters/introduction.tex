%!TEX root = ../thesis-main.tex

\chapter{Einleitung}\label{chapter:introduction}

Ähnlich wie im Abstract soll hier das Problem definiert werden und die Lösung angedeutet werden.
Der Unterschied liegt vor allem in der Ausführlichkeit mit der dies geschieht. 
Das Wichtigste in der Introduction ist die Motivation des Ansatzes: 
\begin{itemize}
	\item \emph{Was ist das Problem?}
	\item \emph{Warum gibt es überhaupt das Problem? }
	\item \emph{Warum will man dies lösen bzw. wieso ist das Problem relevant? }
	\item \emph{Warum ist dein Ansatz notwendig und was soll er leisten (abstrakt)?}
\end{itemize}
In der Forschung gibt es zum Beispiel das CARS-Konzept\footnote{https://libguides.usc.edu/writingguide/CARS}: "Creating a Research Space".
Man geht dabei von einem sehr wichtigen Problem aus und beschreibt, warum es eine ungelöste Nische gibt, die man mit diesem Paper/Arbeit genau füllt.

Es hilft hierbei auch Beispiele zu beschreiben, wenn möglich aus der echten Welt, um dem Leser ein bildhaftes Verständnis der Problematik zu vermitteln.
Am Ende der Introduction sollte die Struktur der restlichen Arbeit eingeführt werden, mit jeweils höchstens einem Satz für jedes Kapitel.

\textbf{Faustregel}: max 10\% Introduction, max 10\% Conclusion


\section{Sonstiges}
\textbf{WICHTIG:} Falls die Thesis ausschließlich digital eingereicht wird (also ohne zusätzliche gedruckte Version), muss in der Datei \texttt{thesis-main.tex} der Befehl \texttt{\textbackslash affidavit} durch \texttt{\textbackslash affidavit[digital]} ersetzt werden.
Dadurch wird die eidesstattliche Erklärung entsprechend angepasst.

Diese Vorlage ist für doppelseitigen Druck eingestellt.
Es ist voreingestellt, dass eine PDF/A-Datei erzeugt wird.
Die beste Kompatibilität hierfür bietet Lua\LaTeX.
Bei anderen Compilern kann dies entsprechend der Informationen in DEMO-TUDaPub zu Problemen führen.
In diesem Fall sollte entweder der Compiler gewechselt oder \texttt{pdfa=false} aktiviert werden.
Dies ist eine Referenz zu Kapitel~\ref{chapter:background}.
Dies ist eine Referenz zu Abschnitt~\ref{sect:dummy-section}.
Dies ist eine Referenz auf eine Quelle~\cite{Luthmann2017}.
Dies ist eine Referenz auf zwei Quellen~\cite{Luthmann2019,Ruland2018}.
Dies ist eine Referenz auf eine Onlinequelle~\cite{parallel-computing}.
Bei Zitationen von Onlinequellen gibt es ein paar Dinge, die beachtet werden müssen:
\begin{itemize}
	\item Zur Literaturverwaltung wird BibTeX verwendet, d.h. es gibt \textbf{keine} \texttt{@online} Typen in der Bibliographie-Datei. Stattdessen verwenden wir \texttt{@misc} und bauen die korrekte Zitierweise per Hand.
	\item Je nachdem, ob in Deutsch oder in Englisch geschrieben wird, muss der String \enquot{Letzter Zugriff:} angepasst werden zu \enquot{Last accessed:}.
	\item Bitte verwende bei den Datumsangaben ausschließlich das ISO-Format (\texttt{\$jahr-\$monat-\$tag}), beispielsweise \texttt{2024-05-07} für den 07. Mai 2024.
\end{itemize}
Dies ist eine Referenz auf eine Dissertation~\cite{tomaszek_stefan_modellbasierte_2021}.



\section{Überschrift eines Abschnitts}\label{sect:dummy-section}

%\lipsum

{\Huge Hier} {\huge steht} {\LARGE ein} {\Large kurzer} {\large Satz} {\normalsize mit} {\small immer} {\footnotesize kleiner} {\scriptsize werdender} {\tiny Schriftgröße}.