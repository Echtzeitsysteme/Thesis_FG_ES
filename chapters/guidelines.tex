\section*{Guidelines für Seminararbeiten und Thesen}

Eingebettet in diese Vorlage findest du eine kurze Anleitung, die dir hilft deinen (ersten) wissenschaftlichen Text zu verfassen.
Dabei kann es sich um ein Paper im Zuge einer Seminararbeit oder auch deine erste Thesis handeln.
Weshalb wir diese Guidelines in Text gegossen haben, ist der einfache Fakt, dass fast jeder zu Anfang dieselben Fehler macht.
Deshalb soll dir dieses Dokument als kleines Nachschlagwerk dienen.
Denke dabei daran diese Teile für die finale Abgabe aus deiner Vorlage zu löschen!

Wichtig: Diese Auflistung erhebt keinen Anspruch auf Vollständigkeit und falls etwas unklar oder unerwähnt bleibt, solltest du dich immer zeitnah an deinen Betreuer wenden.
Auch wären wir über Feedback froh, damit nachfolgende Studenten von dieser Sammlung noch mehr profitieren können. (kontakt(at)es.tu-darmstadt.de) 

Natürlich solltest du auch immer die Richtlinien deines Fachbereichs beachten:
\begin{itemize}
	\item \href{https://www.etit.tu-darmstadt.de/media/etit/studium_1/dokumente_4/richtlinien/richtlinienabschlussarbeiten.pdf}{\color{blue}{Fachbereich Elektrotechnik und Informationstechnik}}
	\item \href{https://www.informatik.tu-darmstadt.de/studium_fb20/im_studium/studienbuero/abschlussarbeiten_fb20/index.de.jsp}{\color{blue}{Fachbereich Informatik}}
\end{itemize}

\section*{Strukturierung}
Fast jede Ausarbeitung ob Paper oder Thesis folgt einer klaren Struktur. 
Es ist an gewissen Stellen sicherlich kein Fehler davon abzuweichen und es kommt auf die Community an, in der man publiziert. 
Allgemein gesprochen gibt es jedoch „konventionsgemäße“ Überschriften:
\begin{itemize}
	\item Abstract / Zusammenfassung
	\item Introduction / Einleitung
	\item Background / Grundlagen
	\item Contribution / Eigener Beitrag
	\item Related Work / Verwandte Arbeiten
	\item Conclusion (= Summary + Outlook) / Abschluss (= Zusammenfassung + Ausblick)
\end{itemize}

Im Folgenden findest du zu jedem dieser Unterpunkte ein paar Erläuterungen.
Lese sie alle durch und verinnerliche sie, bevor du mit dem Aufschreiben anfängst.
Unsere Erfahrung zeigt, dass dies dir viel Arbeit sparen kann!