%!TEX root = ../thesis-main.tex

\chapter{Sonstige Guidelines}
Dieses Kapitel trägt noch einige allgemeinere Tipps zusammen, die dir beim Schreiben helfen sollen.
Obwohl es das letzte Kapitel dieser Vorlage ist und in der Endfassung von dir gelöscht werden sollte, enthält es viele Tipps, die dir das Leben sehr viel einfacher machen werden.
Lies sie deshalb alle durch und versuch beim Schreiben nach diesen "best-practices" vorzugehen.
Denke daran: umso besser deine ersten Versionen sind, desto besser und spezifischer ist das Feedback von deinem Betreuer.

\section{Schreibstil}
Ein schöner Schreibstil ist ebenso wichtig, wie der Inhalt den er beschreibt. 
Ergebnisse bzw. der Weg dahin müssen adäquat beschrieben werden, denn einen Text voller wirrer Sätze möchte keiner lesen und dies schmälert die Arbeit immens.

\begin{itemize}
	\item	Achte auf Rechtschreibfehler und benutze von Anfang an ein Programm mit automatischer Fehlerkorrektur. Eine Endfassung in der zu viele solcher Fehler auftauchen, macht den Eindruck, dass du auf den letzten Meter keine Lust mehr hattest. Wenn dein Betreuer sich vor allem auf Rechtschreib- und Grammatikfehler konzentrieren muss, hat das eventuell negative Auswirkungen auf das inhaltliche Feedback!
	
	\item	Vermeide Schachtelsätze! Ein Komma tut sicherlich keinem weh, aber zu Anfang ist man dazu geneigt es zu übertreiben!
	
	\item	Überlege dir ob du Fachbegriffe, die du erwähnst auch bereits eingeführt hast. Vermeide hierbei, wenn möglich, Vorwärtsreferenzen. Sobald der Leser in deinem Text vor und zurückspringen muss, leidet der Eindruck den er von deinem Text erhält! Diese Vorwärtsreferenzen sind in Ordnung, wenn man ein Fachbuch liest, weil viele unterschiedliche Informationen kreuzweise erklärt werden müssen. Im Allgemeinen sollte darauf jedoch verzichtet werden.
	
	\item	Schreibe nicht zu knapp, aber auch keine Romane. Wenn alles gesagt wurde, braucht man nicht weiter herumzuschwafeln. Das kommt bei Lesern, die aus der Domäne kommen nicht gut an und bringt auch keinen Mehrwert für deinen Text. Wenn dein Text hingegen zu knapp ist, solltest du ihn kritisch betrachten und überlegen, ob genug Informationen hineingeflossen sind, sodass ein (fach-)fremder Leser dir folgen kann. Häufige Kritikpunkte sind hierbei z.B. bei der Beschreibung eines Frameworks, dass der Student schreibt, dass man damit XY macht und das war es. Fragen wie: "Wieso wollen wir sowas überhaupt machen? Was sind Vor- und Nachteile?"  fallen hier gerne unter den Tisch, würden den Text aber bereichern. Ganz allgemein ist die Regel, dass du alles was du schreibst und einführst motivieren solltest.
	
	\item Wie bei vielen Sportarten gilt: Willst du besser werden, umgib dich mit Menschen, die besser sind als du. Schreiben muss wie alles im Leben trainiert werden und durch das Lesen von guten Papern lernt man indirekt auch, wie man seine eigene Sprache verbessern kann oder komplexe Themen adäquat und verständlich präsentiert.
	
	\item Wenn du ein Wort für etwas einführst, dann bleibe auch dabei und springe nicht zwischen Synonymen. Damit wird der Text vielleicht etwas monotoner, aber hier geht es vor allem darum sich präzise auszudrücken. Wenn du anfängst mit Worten zu jonglieren, muss sich der Leser die Frage stellen, ob du vielleicht etwas Neues eingeführt hast, obwohl es sich immer noch um das selbe handelt.
	
	\item Vermeide Umgangssprache! Das können einzelne Wörter sein oder auch Redewendungen. Denke daran, es handelt sich hier um einen seriösen Text!
	
	\item Diskutiere fremde Arbeiten auf eine höfliche Art und Weise!
	
	\item Versuche „wissenschaftliche Wörter“ zu verwenden, wie analysieren, untersuchen, evaluieren etc. Im Internet gibt es ausgiebige Listen dazu.
\end{itemize}

\section{Referenzen}
Referenzen sind ein sehr wichtiger Teil deines Textes. Durch sie stellst du den Zusammenhang zwischen deiner Arbeit und der von anderen Wissenschaftlern her. 
Es geht dabei um Fragestellungen, wie: 
\begin{itemize}
	\item Hat schon einmal jemand etwas Ähnliches gemacht? Wie? Mit welchen Ergebnissen?
	
	\item Gibt es verwandte Ansätze? Wurden diese in Betracht gezogen? Wenn ja, warum? Wenn nein, warum nicht?
	
	\item Woher kommen die Grundlagen für deinen Ansatz?
	
	\item Woher kommen die Werkzeuge?
\end{itemize}

\section{Literaturverzeichnis}
Ans Ende jeder wissenschaftlichen Arbeit gehört ein ordentliches Literaturverzeichnis.
Die Quellen im Literaturverzeichnis sollten sowohl ordentlich, als auch einheitlich dargestellt werden. 
Je nach Arbeit können dazu auch unterschiedliche Details pro Quelle gewünscht sein.
Vor allem wichtig sind hierbei die Autoren, der Titel, wo und wann veröffentlicht wurde (z.B. Konferenzband oder Journal), sowie die Seitenzahlen.
Um das Literaturverzeichnis zu vereinheitlichen, bietet es sich an ein Programm dafür zu nutzen (z.B. JabRef\footnote{Tipp: JabRef hat eine Anbindung an dblp!} oder Citavi).
Unsere Empfehlung ist es außerdem möglichst alle Referenzen aus einer Quelle zu beziehen, wie zum Beispiel dblp. 
Damit sollten die meisten Referenzen bereits einheitlich formatiert sein und die wichtigsten Informationen enthalten.

\section{Plagiarismus}
Plagiarismus ist nicht nur ein Thema in Klausuren. 
Inhalt, der nicht von dir stammt, sollte entsprechend gekennzeichnet werden! 
Im Allgemeinen wirst du auch in den meisten Arbeiten keine Sätze aus anderen Arbeiten finden, sondern eigene Beschreibung des Inhalts mit einer Referenz auf den zugrundeliegenden Text. 
Fachwörter sind dabei natürlich aus dem Originaltext zu übernehmen und müssen nicht durch eine neue Wortneuschöpfung ersetzt werden (siehe Schreibstil).

\section{Sonstiges}
\begin{itemize}
	\item Abbildungen sollten nicht zu groß und nicht zu klein sein. Benutze also keine Bilder um Seiten zu füllen. Dies fällt auf und macht keinen guten Eindruck. Natürlich sollten die Bilder aber trotzdem groß genug sind, dass man sie auch ohne Lupe erkennen und Text darauf lesen kann. Wenn möglich, empfiehlt es sich die Textgrößen so zu wählen, dass sie dem umgebenen Text entsprechen.
	
	\item Benutze Befehle wie \textbf{\textbackslash label\{name\}}, \textbackslash \textbf{cref\{label name\}} und \textbf{\textbackslash cite\{citation key\}} anstatt (1) [3] oder Ähnliches in deinen Text zu schreiben. Sobald du weiter oben noch einmal etwas einfügst, darfst du sonst durch deinen gesamten Text gehen und korrigieren.
	
	\item Schreibe in LaTeX immer 1 Zeile pro Satz! Dies macht deinen LaTeX Code deutlich lesbarer (auch für dich selbst!).
	
	\item Benutze Vektorgraphiken (z.b. svg). Generell gilt es Rastergrafiken (*.png, *.jpp etc.) und (auch hochauflösende) Screenshots zu vermeiden!
	
	\item Teile (mindestens) Kapitel in einzelne Dateien auf und füge sie in der Hauptlatexdatei via \textbf{\textbackslash input\{datei\}} ein, so wie es hier im Template bereits gemacht wird.
	
	\item Tipp: Tools, wie TexStudio oder Grammarly (nur englisch) zeigen Rechtschreibfehler an.
\end{itemize}