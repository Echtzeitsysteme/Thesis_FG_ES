%!TEX root = ../thesis-main.tex

\chapter{Implementierung}\label{chapter:implementation}

Hier wird der eigentliche eigene Beitrag (engl. Contribution) vorgestellt. 
Dies ist für gewöhnlich der Teil an dem man weniger Probleme hat, weil man z.B. durch das Implementieren genau weiß, was man getan hat und diese Dinge auch eine gewisse Ordnung im Kopf haben (Erst habe ich das gemacht, dann das, dann ...).

Wichtig ist hier zu vermeiden Dinge unter den Tisch fallen zu lassen, die einem trivial vorkommen und deshalb unerwähnt bleiben. 
Dem Leser sind diese Dinge womöglich nicht ganz so klar. 
Weiterhin sollte man sich die Frage stellen, wie technisch die Beschreibung sein/werden soll. Es gilt zu vermeiden, dass Implementierungsdetails vom Wesentlichen ablenken und den Leser mit Informationen überfrachten.

Es geht hierbei auch um Konsistenz. 
Wenn in den Grundlagen alles sehr technisch beschrieben ist, aber der Beitrag dann sehr formal ausfällt, wird man nur verwirrte Blicke ernten.
