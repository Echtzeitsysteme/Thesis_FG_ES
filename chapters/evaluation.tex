%!TEX root = ../thesis-main.tex

\chapter{Evaluation}\label{chapter:evaluation}

Allgemeine Struktur einer Evaluation:
\begin{itemize}
	\item Research Questions (RQs) / Forschungsfragen
	\item Experimental Setup / Versuchsaufbau
	\item Metriken(?)
	\item Ergebnisse + Diskussion nach RQ oder insgesamt
	\item Threats to Validity-Diskussion
\end{itemize}
Bei der Evaluation geht es zum Beispiel um die Skalierung eines Ansatzes, aber auch um Dinge wie Zuverlässigkeit, Worst-Case Szenarien, … .
Dies muss man natürlich klar kommunizieren und formuliert deshalb Forschungsfragen, die man mit der Evaluation versucht zu beantworten.

Wichtig ist außerdem, dass der Versuchsaufbau gut beschrieben wird, damit der Leser auch weiß, wie die Ergebnisse zu interpretieren sind. 
Natürlich ist es ebenso wichtig die Ergebnisse selbst gut zu präsentieren. 
Das fängt bereits bei einer eindeutigen Beschriftung der Achsen eines Plots an. 
Auch sollten Datenpunkte und Text gut lesbar sein, aber gleichzeitig nicht zu groß sein (wie mit vielen Dingen sollte man es nicht übertreiben).

Spätestens hier (vielleicht aber auch schon in der Contribution) sollte man präsentieren, wie diese Ergebnisse die Contribution „rechtfertigen“. 
\begin{itemize}
	\item Wie werden die Forschungsfragen durch die Messungen beantwortet?
	\item Sind die Antworten/Messungen wie erwartet?
	\item Wenn ja wieso, wenn nein wieso nicht?
	\item Sind die Ergebnisse wie erhofft?
\end{itemize}
Das Schlimmste was passieren kann ist, dass es \textbf{Interpretationsspielraum} für die Ergebnisse gibt. Ergo sollte man dem Leser die Ergebnisse praktisch „auf die Nase binden“.
